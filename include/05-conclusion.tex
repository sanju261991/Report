%%%%%%%%%%%%%%%%%%%%%%%%%%%%%%%%%%%%%%%%%%%%%%%%%%%%%%%%%%%%%%%%%%%%%%%%%%%%%%%
%
% Conclusion and Future Work
% 
%%%%%%%%%%%%%%%%%%%%%%%%%%%%%%%%%%%%%%%%%%%%%%%%%%%%%%%%%%%%%%%%%%%%%%%%%%%%%%%


\chapter{Conclusion and Future Work}
\label{sec:conclusion_and_future_work}

\section{Summary}

\paragraph{First Chapter - Introduction} contains the introduction to granular material and the Hertz contact theory.

\paragraph{Second Chapter - Simulation Method} contains the introduction to the Finite Element Method and describes the simulation step used for the simulations.

\paragraph{Third Chapter - Results} contains the set of results and their interpretation.  

In this thesis the effects of quasi static approximations in the current models of force contacts and the conversion of kinetic energy to strain energy was studied. After the simulations it was concluded that the effect of the quasi static approximation is not significant as the amount of kinetic energy converted to strain energy is very small and can be neglected. Hence the quasi static approximations are valid.


\section{Future Work}

The conditions of an elastic material and two spheres colliding normally are not possible or improbable in real world scenarios. This work can further be extended to study more real world conditions such as

\subsubsection*{Tangential interaction}

In this study, a normal interaction was considered, but in a real world situations such interactions are rare. Granulates can collide at in any direction and velocity.

\subsubsection*{Non spherical particles}

Granulates generally are not spherical and come in various shapes and sizes. This study can be further extended to granulates of different shapes interacting at different angles and velocities.

\subsubsection*{Viscoelasticity and Plasticity}

This study was focused on purely elastic materials, but not all matter found in nature can be considered as elastic. Further studies can be conducted considering viscoelastic and plastic materials.

\subsubsection*{Multi-particle Interaction}

Granular materials such as sand, snow and powders can consist of thousands of granulates and in scenarios such as landslides and snow storms can have collisions from multiple particles at a time.