%%%%%%%%%%%%%%%%%%%%%%%%%%%%%%%%%%%%%%%%%%%%%%%%%%%%%%%%%%%%%%%%%%%%%%%%%%%%%%%
%
% Evaluation
% 
%%%%%%%%%%%%%%%%%%%%%%%%%%%%%%%%%%%%%%%%%%%%%%%%%%%%%%%%%%%%%%%%%%%%%%%%%%%%%%%


\chapter{Results}
\label{sec:results}

\section{Results}

\newpage

\begin{figure}[H]
\centering
\subfloat[Velocity 0.01m/s]{
\includegraphics[width=0.55\textwidth]{{../images/deformationVStime/Velocity1.0}.png}
\label{fig:def1}
}
\subfloat[Velocity 0.10m/s]{
\includegraphics[width=0.55\textwidth]{{../images/deformationVStime/Velocity10.0}.png}
\label{fig:def10}
}

\subfloat[Velocity 0.20m/s]{
\includegraphics[width=0.55\textwidth]{{../images/deformationVStime/Velocity20.0}.png}
\label{fig:def20}
}
\subfloat[Velocity 0.30m/s]{
\includegraphics[width=0.55\textwidth]{{../images/deformationVStime/Velocity30.0}.png}
\label{fig:def30}
}

\subfloat[Velocity 0.50m/s]{
\includegraphics[width=0.55\textwidth]{{../images/deformationVStime/Velocity50.0}.png}
\label{fig:def50}
}
\subfloat[Velocity 3.0m/s]{
\includegraphics[width=0.55\textwidth]{{../images/deformationVStime/Velocity300.0}.png}
\label{fig:def300}
}
\caption{Displacement of the center of the sphere for various impact velocities}
\label{fig:def}
\end{figure}

The plots in \ref{fig:def} show displacement of the center of the sphere with respect to time. The plots shows that as the impact velocities are higher, the difference between the simulation data and the theoretical data is more visible. This errors is due the quasi static assumptions.

\begin{figure}[H]
\centering
\subfloat[Velocity 0.01m/s]{
\includegraphics[width=0.55\textwidth]{{../images/force/Force-vel1.0}.png}
\label{fig:force1}
}
\subfloat[Velocity 0.10m/s]{
\includegraphics[width=0.55\textwidth]{{../images/force/Force-vel10.0}.png}
\label{fig:force10}
}

\subfloat[Velocity 0.20m/s]{
\includegraphics[width=0.55\textwidth]{{../images/force/Force-vel20.0}.png}
\label{fig:force20}
}
\subfloat[Velocity 0.30m/s]{
\includegraphics[width=0.55\textwidth]{{../images/force/Force-vel30.0}.png}
\label{fig:force30}
}

\subfloat[Velocity 0.50m/s]{
\includegraphics[width=0.55\textwidth]{{../images/force/Force-vel50.0}.png}
\label{fig:force50}
}
\subfloat[Velocity 3.0m/s]{
\includegraphics[width=0.55\textwidth]{{../images/force/Force-vel300.0}.png}
\label{fig:force300}
}
\caption{Contact force vs Time for various impact velocities}
\label{fig:force}
\end{figure}

The plots in \ref{fig:def} show the contact force between the sphere and the rigid plane. The plots shows that as the impact velocities are higher, the difference between the simulation data and the theoretical data is more visible. This errors is due the quasi static assumptions.



\section{COR}

\begin{figure}[H]
\includegraphics[width=1.0\textwidth]{../images/COR/COR.png}
\caption{COR}
\label{fig:COR}
\end{figure}

The figure \ref{fig:COR} shows the co-efficient of restitution vs various impact velocities. We can see that the co-efficient of restitution increases as the impact velocities are increased. The bumps in the plot can be explained after performing a Fourier analysis on the models. The Fourier analysis shows that the bumps correspond to the 


\section{Parametric Study}

\subsection{Different Youngs Modulus}

\begin{figure}[H]
\subfloat[COR]{
\includegraphics[width=1.0\textwidth]{../images/parametricStudy/COR.png}
\label{fig:CORdiffE}
}
\end{figure}


\begin{figure}[H]
\subfloat[COR High Velocity]{
\includegraphics[width=0.55\textwidth]{../images/parametricStudy/COR_higherVEL.png}
\label{fig:CORdiffEHigh}
}
\subfloat[COR Low Velocity]{
\includegraphics[width=0.55\textwidth]{../images/parametricStudy/COR_lowerVEL.png}
\label{fig:CORdiffELow}
}
\end{figure}

\subsection{Different Diameters}

\begin{figure}[H]
\includegraphics[width=1.0\textwidth]{../images/parametricStudy/CORvsVELdiffDAI.png}
\caption{COR}
\label{fig:CORDiffDia}
\end{figure}


