%%%%%%%%%%%%%%%%%%%%%%%%%%%%%%%%%%%%%%%%%%%%%%%%%%%%%%%%%%%%%%%%%%%%%%%%%%%%%%%
%
% Background
% 
%%%%%%%%%%%%%%%%%%%%%%%%%%%%%%%%%%%%%%%%%%%%%%%%%%%%%%%%%%%%%%%%%%%%%%%%%%%%%%%


\chapter{Background}
\label{sec:background}

\blindtext[3][2]

\begin{figure}[ht]
	\centering
	\includegraphics[width=0.92\textwidth]{screenshot}
	\caption{Some sample caption}
	\label{fig:cbc_enc}
\end{figure}


In Figure~\ref{fig:cbc_enc} you can see how to refer to figures in text. \blindtext{}




\begin{figure}
\centering
\footnotesize
\begin{minipage}[b]{0.50\textwidth}
\centering
\begin{alignat*}{3}
	\hat B_0     & := &\;& IP(P)           \\
	\hat B_{i+1} & := &&   R_i(\hat B_i)   \\
	C            & := &&   FP(\hat B_{32}) \\
	\text{where~~~~~~~~~~~~~~~~~~~~~}      \\
	R_i(X)       & =  &&   L(\hat S_i(X \oplus \hat K_i))
						&\qquad& i = 0, \ldots, 30 \\
	R_i(X)       & =  &&   \hat S_i(X \oplus \hat K_i) \oplus \hat K_{32}
						&& i = 31
\end{alignat*}
\caption{Left part of a complex figure}
\label{fig:serpentcode}
\end{minipage}
\hspace{0.25cm}
\vline
\hspace{0.25cm}
\begin{minipage}[b]{0.40\textwidth}
\centering
\begin{align*}
	X_0,X_1,X_2,X_3 & := \hat S_i(\hat B_i \oplus \hat K_i) \\
	X_0             & := X_0 <<< 13 \\
	X_2             & := X_2 <<< 3 \\
	X_1             & := X_1 \oplus X_0 \oplus X_2 \\
	X_3             & := X_3 \oplus X_2 \oplus (X_0 << 3) \\
	X_1             & := X_1 <<< 1 \\
	X_3             & := X_3 <<< 7 \\
	X_0             & := X_0 \oplus X_1 \oplus X_3 \\
	X_2             & := X_2 \oplus X_3 \oplus (X_1 << 7) \\
	X_0             & := X_0 <<< 5 \\
	X_2             & := X_2 <<< 22 \\
	\hat B_{i+1}    & := X_0,X_1,X_2,X_3
\end{align*}
\caption{Right part of the figure}
\label{fig:serpentlin}
\end{minipage}
\end{figure}

A more complex figure is shown in Figure~\ref{fig:serpentlin}. \blindtext





\begin{figure}
\centering
\begin{minipage}[b]{0.45\textwidth}
\centering
\begin{tabular}{c}
\begin{lstlisting}
  unsigned char s0[16] = {
          3,  8, 15,  1,
         10,  6,  5, 11,
         14, 13,  4,  2,
          7,  0,  9, 12
  };
\end{lstlisting}
\end{tabular}
\caption{Serpent S-box $S_0$ written as array}
\label{fig:serpents0a}
\end{minipage}
\hspace{0.25cm}
\vline
\hspace{0.25cm}
\begin{minipage}[b]{0.45\textwidth}
\centering
\begin{tabular}{c}
\begin{lstlisting}
#define S0(x0, x1, x2, x3, x4) ({ \
                        x4  = x3; \
  x3 |= x0;  x0 ^= x4;  x4 ^= x2; \
  x4 = ~x4;  x3 ^= x1;  x1 &= x0; \
  x1 ^= x4;  x2 ^= x0;  x0 ^= x3; \
  x4 |= x0;  x0 ^= x2;  x2 &= x1; \
  x3 ^= x2;  x1 = ~x1;  x2 ^= x4; \
  x1 ^= x2;                       \
})
\end{lstlisting}
\end{tabular}
\caption{$S_0$ written as logical sequence}
\label{fig:serpents0l}
\end{minipage}
\end{figure}


A figure using listings is shown in Figure~\ref{fig:serpents0l}. \blindtext

As an example of a complex enumeration, here is
the kernel tree of Linux:
%
\begin{itemize}
	\item {\bf\code{arch/x86}}: x86\_32 and x86\_64 specific source code
	\begin{itemize}
		\item {\bf\code{crypto}}: x86 specific implementation of ciphers
		\item {\bf\code{include/asm}}: x86 specific kernel headers
	\end{itemize}
	\item \code{block}: Block I/O layer
	\item {\bf\code{crypto}}: Crypto API
	\item \code{drivers}: Device drivers
	\item \code{firmware}: Device firmware
	\item \code{fs}: Filesystem implementations
	\item {\bf\code{include}}: Kernel headers
	\begin{itemize}
		\item {\bf\code{crypto}}: Crypto API headers
	\end{itemize}
	\item \code{init}: Kernel boot and initialization code
	\item \code{ipc}: Interprocess communication
	\item \code{kernel}: Core subsystems (e.g. scheduling)
	\item \code{lib}: Helper routines
	\item \code{mm}: Memory Management subsystem
	\item \code{net}: Networking subsystem (Ethernet, IPv4, IPv6, ...)
	\item \code{security}: Linux Security Module
	\item \code{sound}: Sound subsystem
	\item \code{virt}: Virtualization infrastructure
\end{itemize}

\Blindtext[2][1]