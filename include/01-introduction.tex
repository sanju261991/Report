%%%%%%%%%%%%%%%%%%%%%%%%%%%%%%%%%%%%%%%%%%%%%%%%%%%%%%%%%%%%%%%%%%%%%%%%%%%%%%%
%
% Introduction
% 
%%%%%%%%%%%%%%%%%%%%%%%%%%%%%%%%%%%%%%%%%%%%%%%%%%%%%%%%%%%%%%%%%%%%%%%%%%%%%%%


\chapter{Introduction}

Some intro

\section{Granulates}

A granular material is a collection of distinct macroscopic particles, such as sand in an hourglass or peanuts in a container.  The evolution of the particles follows Newton's equations, with repulsive forces between particles that are non-zero only when there is a contact between particles.  Although granular materials are very simple to describe they exhibit a tremendous amount of complex behavior, much of which has not yet been satisfactorily explained.  They behave differently than solids, liquids, and gases which has led many to characterize granular materials as a new form of matter. 

\section{Particle Simulations of Granulates}


\section{Particle Models}


\section{Aims}


\section{Acknowledgments}

A big thank you for the support to Dr.Patric Mueller

