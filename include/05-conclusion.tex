%%%%%%%%%%%%%%%%%%%%%%%%%%%%%%%%%%%%%%%%%%%%%%%%%%%%%%%%%%%%%%%%%%%%%%%%%%%%%%%
%
% Conclusion and Future Work
% 
%%%%%%%%%%%%%%%%%%%%%%%%%%%%%%%%%%%%%%%%%%%%%%%%%%%%%%%%%%%%%%%%%%%%%%%%%%%%%%%


\chapter{Conclusion and Future Work}
\label{sec:conclusion_and_future_work}

\section{Summary}

In this thesis the validity of the current force model used in simulation of granular materials was challenged. The current theory for contact problems for elastic and viscoelastic materials involve quasi static approximations. The quasi static assumptions imply that the characteristic deformation rate is much smaller than the speed of sound in the material and that the relaxation time of the particle's material is negligible compared to the duration of the contact. The validity of these assumptions is probed by considering the case of impact between two elastic spheres. The simulations were preformed using the Finite Element tool Abaqus CAE. 

To validate the quasi static assumptions various quantities such as deformation, contact force and coefficient of restitution were studied. The plots of deformation do show an error between the theory and the simulation for higher impact velocities, but a more convincing quantity to determine the effect of the assumptions was the coefficient of restitution, which was calculated using the kinetic energy of the sphere before and after the impact. These results show the loss of kinetic energy in the sphere, which was converted to strain energy resulting in vibrations within the sphere. The loss of kinetic observed in the simulations was much smaller than expected, hence the quasi static assumption are valid. A modal analysis was also conducted to further verify the nature of the coefficient of restitution vs impact velocity plot, which are affected by the natural vibrations of the sphere. 

A parametric study was also conducted by varying the Young's Modulus and diameter of sphere. In the case of the varying Young's Modulus, the coefficient of restitution seems to reach a maximum value for higher Young's Moduli, whereas the variation of the radius of the sphere did not results in any change in the plots, which was unexpected as for viscoelastic materials the coefficient of restitution is inversely proportional to the radius of the sphere. 


\section{Future Work}

The conditions of an elastic material and two spheres colliding normally are not possible or improbable in real world scenarios. This work can further be extended to study more real world conditions such as

\subsubsection*{Tangential interaction}

In this study, a normal interaction was considered, but in a real world situations such interactions are rare. Granulates can collide at in any direction and velocity.

\subsubsection*{Non spherical particles}

Granulates generally are not spherical and come in various shapes and sizes. This study can be further extended to granulates of different shapes interacting at different angles and velocities.

\subsubsection*{Viscoelasticity and Plasticity}

This study was focused on purely elastic materials, but not all matter found in nature can be considered as elastic. Further studies can be conducted considering viscoelastic and plastic materials.

\subsubsection*{Multi-particle Interaction}

Granular materials such as sand, snow and powders can consist of thousands of granulates and in scenarios such as landslides and snow storms can have collisions from multiple particles at a time.