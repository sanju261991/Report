%%%%%%%%%%%%%%%%%%%%%%%%%%%%%%%%%%%%%%%%%%%%%%%%%%%%%%%%%%%%%%%%%%%%%%%%%%%%%%%
%
% Introduction
% 
%%%%%%%%%%%%%%%%%%%%%%%%%%%%%%%%%%%%%%%%%%%%%%%%%%%%%%%%%%%%%%%%%%%%%%%%%%%%%%%


\chapter{Introduction}

Some intro

\section{Granulates}

A granular material is a conglomeration of discrete solid, macroscopic particles characterized by a loss of energy whenever the particles interact (the most common example would be friction when grains collide). The constituents that compose granular material must be large enough such that they are not subject to thermal motion fluctuations. Thus, the lower size limit for grains in granular material is about 1$\mu m$. On the upper size limit, the physics of granular materials may be applied to ice floes where the individual grains are icebergs and to asteroid belts of the Solar System with individual grains being asteroids. \citep{duran}

\section{Particle Simulations of Granulates}


\section{Particle Models}


\section{Aims}


\section{Acknowledgements}

A big thank you for the support to Dr.Patric Mueller

