%%%%%%%%%%%%%%%%%%%%%%%%%%%%%%%%%%%%%%%%%%%%%%%%%%%%%%%%%%%%%%%%%%%%%%%%%%%%%%%
%
% Implementation
% 
%%%%%%%%%%%%%%%%%%%%%%%%%%%%%%%%%%%%%%%%%%%%%%%%%%%%%%%%%%%%%%%%%%%%%%%%%%%%%%%


\chapter{Implementation}
\label{sec:implementation}


\Blindtext[5][2]

\begin{figure}
\centering
\begin{tabular}{c}
\begin{lstlisting}
static int __init serpent_init(void)
{
    u64 xcr0;
    if (!cpu_has_avx || !cpu_has_osxsave) {
        printk(KERN_INFO "AVX instructions are not detected.\n");
        return -ENODEV;
    }
    xcr0 = xgetbv(XCR_XFEATURE_ENABLED_MASK);
    if ((xcr0 & (XSTATE_SSE | XSTATE_YMM)) != (XSTATE_SSE | XSTATE_YMM)) {
        printk(KERN_INFO "AVX detected but unusable.\n");
        return -ENODEV;
    }
    return crypto_register_algs(serpent_algs, ARRAY_SIZE(serpent_algs));
}

static void __exit serpent_exit(void)
{
    crypto_unregister_algs(serpent_algs, ARRAY_SIZE(serpent_algs));
}

module_init(serpent_init);
module_exit(serpent_exit);

MODULE_DESCRIPTION("Serpent Cipher Algorithm, AVX optimized");
MODULE_LICENSE("GPL");
MODULE_ALIAS("serpent");
\end{lstlisting}
\end{tabular}
\caption{Serpent AVX module initialization}
\label{fig:serpent_init}
\end{figure}

Some complex code is shown in Figure~\ref{fig:serpent_init}. \blindtext


