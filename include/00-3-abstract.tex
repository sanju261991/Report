%%%%%%%%%%%%%%%%%%%%%%%%%%%%%%%%%%%%%%%%%%%%%%%%%%%%%%%%%%%%%%%%%%%%%%%%%%%%%%%
%
% Abstract
% 
%%%%%%%%%%%%%%%%%%%%%%%%%%%%%%%%%%%%%%%%%%%%%%%%%%%%%%%%%%%%%%%%%%%%%%%%%%%%%%%

% Pseudo chapter
\chapter*{\ }


\vspace{0.75em}

\vspace{2em}
\begin{center}
	\begin{large}
		\textbf{Abstract}
	\end{large}
\end{center}
\vspace{0.75em}

An important model for granular particles are elastic and viscoelastic spheres. The macroscopic interaction forces for such objects are commonly obtained from the continuum mechanical equations of motion for elastic and viscoelastic material in quasi static approximation. The same holds true for the coefficients of restitution of colliding spheres which are, in turn, obtained from the macroscopic interaction forces. The quasi static assumption implies that the characteristic deformation rate is much smaller than the speed of sound in the material and that the relaxation time of the particle's material is negligible compared to the duration of the contact.  In this work the validity of these assumptions is probed for realistic impact scenarios by comparing to a direct numerical solution of the underlying continuum mechanical equations of motion by means of finite elements.   


\newpage

\chapter*{\ }


\vspace{0.75em}

\vspace{2em}
\begin{center}
	\begin{large}
		\textbf{Acknowledgments}
	\end{large}
\end{center}
\vspace{0.75em}



I would like to express my sincere gratitude to my advisor Dr. Patric Müller and Prof. Dr. Thorsten Pöschel for their continuous support, patience, motivation, and immense knowledge. Their guidance helped me in all the time of research and writing of this thesis.

I would also like to thank my family and friends, for all their love and support.